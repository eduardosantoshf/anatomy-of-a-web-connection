\documentclass{article}
% Margin definition.
\usepackage[a4paper,total={6.8in, 8.5in}]{geometry}
\usepackage{parskip}
% Images.
\usepackage{graphicx}
\usepackage[hidelinks, bookmarks=true]{hyperref}
% Encoding.
\usepackage[english]{babel}
\usepackage[utf8]{inputenc}
% Helvetic font.
\usepackage[scaled]{helvet}
\renewcommand\familydefault{\sfdefault} 
% Header for UA logo.
\usepackage{fancyhdr}
% Dots in index.
\usepackage[titles]{tocloft}
\renewcommand{\cftsubsecleader}{\Large\cftdotfill{0}}
\renewcommand{\cftsecleader}{\Large\cftdotfill{0}}
\renewcommand{\cftsecfont}{\large\bfseries\scshape}
\renewcommand{\cftsubsecfont}{\scshape}
\renewcommand*{\HyperDestNameFilter}[1]{\jobname-#1}
% Dot after number in (sub)sections and in toc.
\renewcommand{\cftsecaftersnum}{.}
\renewcommand{\cftsubsecaftersnum}{.}
\usepackage{titlesec}
\titlelabel{\hspace{-0.5cm}\quad}
\usepackage[letterspace=45]{microtype}
\newcommand*{\fullref}[1]{\hyperref[{#1}]{\autoref*{#1} \nameref*{#1}}}
% Header with UA logo definition. 
\pagestyle{fancy}
\fancyhf{}
\chead{
    \includegraphics[width=5in]{./images/header_ua.png}
}
\setlength\headheight{20pt}
% Footer with page number.
\rfoot{Page \thepage}
\renewcommand{\footrulewidth}{0.1pt}
% Rename table of contents title to "Index"
\renewcommand{\contentsname}{\normalsize Index \vspace{0.6cm}}
% Water mark
\newsavebox\mybox
\usepackage[printwatermark]{xwatermark}
\usepackage{xcolor}
\usepackage{tikz}
% paragraph
\newcommand\tab[1][1cm]{\hspace*{#1}}
\setlength\parindent{24pt}
%images
 \usepackage{graphicx}
\usepackage{caption}
% footnotes at bottom
\usepackage[bottom]{footmisc}
% Urls with line break
\usepackage{pdflscape}
% Drawing functions
\usepackage{tikz}
\usepackage{pgfplots}
\pgfplotsset{width=7cm, height=4cm, compat=1.17}

\usepackage{multicol}
\setlength{\columnsep}{1cm}

%%%%%%%%%%%% References/Bibliography %%%%%%%%%%%%
\usepackage{biblatex}
\addbibresource{bibliography.bib}

%%%%%%%%%%%%%%%%%%%%%%%%%%%%%%%%%%%%%%%%%%%%%%%%%

\begin{document}

%%%%%%%%%%%%%%%%%% Cover Page %%%%%%%%%%%%%%%%%%
\title{\vspace{-0.9cm}
       \vspace{1cm}
       \normalsize
       \raggedright\textbf{Title: \hspace{1.5cm} Anatomy of a Web Connection: A Brief Analysis} \\ \vspace{0.4cm}
       \raggedright\textbf{Autor: \hspace{1.3cm} Eduardo Santos} \\ \vspace{0.4cm}
       \raggedright\textbf{Date: \hspace{1.45cm} 21/03/2021} \\}
\author{}
\date{}

\maketitle
\thispagestyle{fancy}

%%%%%%%%%%%%%%%%%% END Cover Page %%%%%%%%%%%%%%%%%%

\vspace{-1.4cm}

\tableofcontents


\fontsize{10pt}{13pt}
\selectfont
\lsstyle

\titlelabel{\thetitle.\quad}	

\section{Introductory Note}

\tab This assignment consists in the use of the "traceroute" command followed by the domain \textbf{"www.cmu.edu"}, and interpreting the results obtained.  

\noindent The specific objectives of this assignment are the following:

\begin{itemize}
    \item To provide a plausible identification of the technologies, processes, actors and business models involved in an web connection.
    \item To identify possible social and economic implications associated with the identified technologies, processes, actors and business models.
\end{itemize}

\section{Summary / Abstract}

\tab This assignment addresses things that can happen during a web connection.

Using the "traceroute" command, we will be able to analyze how a web connection works, and the path that the packages take in that connection.
From that analysis, we will be able to reach a conclusion about the previously referred paths, and the hops of the connection, referring also the players involved in each hop.

This assignment will also contemplate some other points, such as the operations, processes, techniques and technologies involved in each step, trying to situate these in the framework of the ISO OSI model, as well as some possible social and economic implications triggered by the point previously mentioned.

\section{Framework}

The web as a major importance in our lives, without it, most of our daily common tasks, tasks that we consider easy, wouldn't be so easy as we wished to. 

We can find an entire "world" inside our computers, smartphones, etc. This is only possible because there are many technologies, processes and actors involved, all of them through an abstract layer that hides the real complexity of the whole system. The operations inside that layer might have significant social and economic implications.

\section{Components involved in a web connection}

\subsection{Technologies}

\subsubsection{TCP/IP}




% No cite makes all references appear, even if there's no citation on the text
\nocite{*}
\printbibliography

\end{document}